\documentclass[11pt,a4paper]{article}
\usepackage{float}
\usepackage[UTF8]{ctex} % ✅ 中文支持
\usepackage{geometry}
\usepackage{fancyhdr}
\usepackage{titlesec}
\usepackage{graphicx}
\usepackage{enumitem}
\usepackage{amsmath,amssymb}
\usepackage{xcolor}
\usepackage{tcolorbox} % ✅ 灰底支持
\usepackage{pgffor} % ✅ 自动导入题目
\usepackage{xstring}
\usepackage[colorlinks=true, linkcolor=blue, urlcolor=cyan, citecolor=green]{hyperref}
\usepackage{circledtext}

\geometry{left=1.8cm,right=1.8cm,top=2cm,bottom=2cm}
\setlength{\headheight}{14pt}
\pagestyle{fancy}
\fancyhf{}
\fancyhead[C]{第二十届“程协杯”程序设计竞赛}
\fancyfoot[C]{Page \thepage}

\setlist[itemize]{
	noitemsep,          % 取消条目间的额外间距
	topsep=4pt,         % 与上方文字的间距
	partopsep=0pt,      % 当在段落顶部时的额外间距
	parsep=0pt,         % 条目内段落间距
	leftmargin=*,       % 左对齐
	labelsep=0.5em      % 标签与文本的间距
}

% ======= 带手动缩进命令的伪代码环境 =======
\newenvironment{pseudocode}{
	\begin{tcolorbox}[
		colback=gray!10,
		colframe=gray!50,
		boxrule=0pt,
		sharp corners,
		left=4pt, right=4pt, top=4pt, bottom=4pt,
		fontupper=\ttfamily\small
		]
		\obeylines
		\obeyspaces
	}{
	\end{tcolorbox}
}

% 缩进命令
\newcommand{\indentcode}{\hspace*{2em}}

% ======= Problem 环境 =======
\newcounter{problem}
\renewcommand{\theproblem}{\Alph{problem}}
\newenvironment{problem}[1][]{
	\clearpage  % 每题从新页开始
	\refstepcounter{problem}
	\setcounter{examplecounter}{0} % 重置示例计数器
	\section*{Problem \theproblem. #1}
	\addcontentsline{toc}{section}{Problem \theproblem. #1} % 加入目录
	\vspace{0.4em}\hrule\vspace{0.6em}
}{\vspace{0.6em}\hrule\vspace{1em}}

% ======= 定义 Example 计数器 =======
\newcounter{examplecounter}
\setcounter{examplecounter}{0}

% ======= Example 环境(自动编号)=======
\newenvironment{examplebox}[2]{
	\stepcounter{examplecounter}
	\vspace{0.5em}
	\begin{tcolorbox}[colback=gray!10,boxrule=0pt,sharp corners,arc=1pt,left=4pt,right=4pt,top=4pt,bottom=4pt]
		\begin{minipage}[t]{0.48\textwidth}
			\textbf{Input \theexamplecounter}\\
			\texttt{#1}
		\end{minipage}
		\hfill
		\begin{minipage}[t]{0.48\textwidth}
			\textbf{Output \theexamplecounter}\\
			\texttt{#2}
		\end{minipage}
	\end{tcolorbox}
}{}

\begin{document}
	
	% ---- 封面 ----
	\begin{titlepage}
		\centering
		\vspace*{2cm}
		\includegraphics[width=0.25\textwidth]{logo.png}\par
		\vspace{1cm}
		{\Huge 第二十届“程协杯”程序设计竞赛\par}
		\vspace{0.8cm}
		{\Large 2025 年 11 月 16 日,成都理工大学\par}
		\vfill
		{\large Produced by 第二十届“程协杯”命题组}\\[0.3cm]
		{\large \today}
	\end{titlepage}
	
	\tableofcontents
	\clearpage
	
	% ---- 导入所有题目 ----
	\IfFileExists{problems/A.tex}{\begin{problem}[对称]
	埃及法老Aiji Falao是一位暴君,他喜欢抓人来解决自己出的谜题,成功解决能获得一定的赏金,但如果无法解决就会被当作祭品。
	很不幸,这天,你被抓住了;但是很幸运,你的大脑连接到了第114154届程协杯比赛现场的某台电脑上。
	
	Aiji Falao向你发起了询问:
	
	\begin{quote}
		我有一个$n\times n$的棋盘和$n$个相同的棋子。每一行、每一列最多放一个棋子,而且棋子摆放的形状必须关于主对角线(从左上到右下)对称。告诉我,棋子有多少种摆放的方式?
	\end{quote}
	
	为了避免成为祭品,你需要立即编程计算出有多少种摆法。
	
	\subsection*{输入描述}
	第一行一个整数 $T(1\le T\le 10^6)$ 表示数据组数。
	
	接下来 $T$ 行,每行一个整数 $n(1\le n\le 10^6)$ 表示棋盘大小和土块个数。
	
	\subsection*{输出描述}
	对于每组数据,输出一行一个整数,表示摆法的数量。
	
	由于答案可能非常大,所以答案对 $10^9+7$ 取模。
	
	\subsection*{样例}
	\begin{examplebox}{
		3\\
		1\\
		2\\
		3
	}{
		1\\
		2\\
		4
	}
	\end{examplebox}
	
	\subsection*{说明/提示}
	如果存在位置$(i,j)$,使得两种摆法在$(i,j)$上一个有棋子而另一个没有,那么这两种摆法是不同的。
	
	当$n=1$时,摆法有$(1,1)$。
	
	当$n=2$时,摆法有$(1,1)(2,2)$、$(1,2)(2,1)$。
	
	当$n=3$时,摆法有$(1,1)(2,2)(3,3)$、$(1,1)(3,2)(2,3)$、$(1,3)(2,2)(3,1)$、$(1,2)(2,1)(3,3)$。
\end{problem}}{}
	\IfFileExists{problems/B.tex}{\begin{problem}[拼手速]
	
	6111年52月02日, 小M参加了第114514届程协杯。
	这次程协杯持续时间为 $10^{10^{10}}$ 分钟(以下默认时间单位为分钟),有 $n$ 道题目,采用CIALLO赛制。
	
	CIALLO赛制和大家熟知的ACM赛制相似,不同的是,CIALLO赛制没有罚时!
	每道题目的用时为比赛开始到该题目被正确解决的时间,最终,选手的总用时是所有被正确解决的题目的用时的和。
	
	每天刷水题的小M瞬间看完了所有的题目,并且算出了每道题目要花费的时间。他立刻开始解题,每道题目解出后立刻开始解下一题,保证不会出错。
	
	小M可以以任意顺序解题,请求出小M的总用时最低是多少。
	
	\subsection*{输入描述}
	第一行一个整数 $T(1\leq T\leq 100)$ 表示数据组数,接下来按组给出数据。
	
	对于每组数据:
	
	\begin{quote}
		第一行一个整数 $n(1\leq n\leq20)$ 代表程协杯的题目数量;
		
		随后一行 $n$ 个整数 $t_1,t_2......t_n(0\leq t_i\leq 1000)$,$t_i$代表道第$i$道题目花费的时间。
	\end{quote}
	
	\subsection*{输出描述}
	对于每组数据,输出一行一个整数,代表小M最低的总用时。
	
	\subsection*{样例}
	\begin{examplebox}{
		2\\
		2\\
		2 1\\
		5\\
		1 1 1 1 1
	}{
		4\\
		15
	}
	\end{examplebox}
	
	\subsection*{说明/提示}
	对于第一组数据:最优的解题顺序是$[2,1]$,用时为$[1,3]$,总用时为4。
	
	对于第二组数据:解题顺序是$[1,2,3,4,5]$,用时为$[1,2,3,4,5]$,总用时为15。
	
	\textbf{显然,选手应当先解决花费时间少的题目。}
	
\end{problem}}{}
	\IfFileExists{problems/C.tex}{\begin{problem}[非成都电路板]
	6Jun 来到成都,打算在这座慢节奏又卷得离谱的城市里搞点新东西。
	他正调试一块电路板。电路板可以视为由芯片组成的$n \times m$的矩阵 $A$,第$i$行第$j$个芯片的参数为$A_{i,j}$。但是问题来了:
	
	\begin{quote}
		6Jun 讨厌一切\textbf{「同」}。
	\end{quote}
	
	同样的想法、同样的咖啡、同样的 Bug、还有最要命的 —— 相邻芯片的值也\textbf{「同」}!
	一旦两块相邻的芯片参数一样,整个系统就陷入“同化危机”: 电路发烫、风扇狂转。这个危机会让6Jun直接红温!
	
	为了避免出现“同化危机”,6Jun 决定动手修改芯片参数。他可以进行一次如下操作:选择一些芯片,让这些芯片的参数 $+x$。但是,操作只能进行一次,于是他找到了聪明的你并且希望你能够帮他解决这个难题。
	形式化地,你需要设计一个新的矩阵 $B$,让它符合以下要求:
	
	\begin{quote}
	对于所有的位置 $(i,j)$,$B_{i,j} = A_{i,j}$或$B_{i,j} = A_{i,j} + x $,其中$1\leq i\leq n,1\leq j\leq m$。
	
	不存在位置 $(i,j)$ 使得 $B_{i,j}=B_{i+1,j}$ 或 $B_{i,j}=B_{i,j+1}$,其中$1\leq i<n,1\leq j<m$。\footnote{当所有芯片都各有个性、再也没有\textbf{「同」}的那一刻,6Jun 坐在成都的夜色里,喝着蜜雪冰城,露出满意的笑:“哈哈哈,终于没有‘同’了。”}
	\end{quote}
	
	\subsection*{输入描述}
	
	第一行包一个整数 $T$ $(1 \leq T \leq 100)$ 代表数据组数,接下来按组给出数据。
	
	对于每组数据:
	
	\begin{quote}
		第一行三个整数 $n , m, x$ $(1 \leq n , m \leq 100, 1\leq x\leq 10^4)$ 分别表示行数、列数、修改值。
		
		接下来 $n$ 行,每行 $m$ 个整数,第$i$行第$j$个表示 $ A_{i,j} (0\leq A_{i,j}\leq10^8)$。
	\end{quote}
	
	\subsection*{输出描述}
	对于每组数据,输出 $n$ 行,每行 $m$ 个整数。 第 $i$ 行第 $j$ 个表示 $ B_{i,j} $。
	
	\subsection*{样例}
	\begin{examplebox}{
		2\\
		1 6 1\\
		3 2 1 1 2 3\\
		3 2 1\\
		2 2\\
		3 3\\
		4 4
	}{
		4 3 2 1 2 3\\
		2 3\\
		3 4\\
		4 5
	}
	\end{examplebox}
	
	\subsection*{说明/提示}
	对于第一组数据:选择$(1,1),(1,2),(1,3)$。
	对于第二组数据:选择$(1,2),(2,2),(3,2)$。
\end{problem}}{}
	\IfFileExists{problems/D.tex}{\begin{problem}[线段树?]
	小曾和小张是两名数学爱好者。小曾有一个正整数 $X$,小张有一个长度为 $n$ 的序列 $a$。
	现在,他们决定测试一下他们之间的默契度。小曾提出了这样一种方式:
	
	\begin{quote}
		每次从小张的序列中随机选择一个区间$l,r$并计算区间中所有元素的乘积 $val_{l,r}$,然后找到最大非负整数$k$,使得小曾手中的整数$X$的$k$次方能够整除$val_{l,r}$。将$k$记为本次的得分。
	\end{quote}
	
	现在,他们找到了作为编程高手的你来帮他们计算得分并输出。形式化地,对于每次询问的$l,r$,求出最大的非负整数$k$使得:
	
	$$
	(\prod_{i=l}^ra_i) \mod X^k =0
	$$
	
	\subsection*{输入描述}
	第一行两个整数 $n,X(1\le n\le10^5,2\le X\le 10^{9})$ 表示小张的序列的元素的个数和小曾的正整数。
	
	第二行 $n$ 个整数$a_1,a_2......a_n(1\le a_i\le 10^9)$表示小张的序列 $a$。
	
	第三行一个整数 $m(1\le m\le 10^5)$ 表示询问的次数。
	
	接下来 $m$ 行每行两个整数 $l,r(1\leq l\leq r\leq n)$ 表示询问的区间。
	
	\subsection*{输出描述}
	对于每个询问,输出一行一个整数表示这个询问的得分 $k$。
	
	\subsection*{样例}
	\begin{examplebox}{
		3 114514\\
		114514 10 114514\\
		2\\
		1 3\\
		2 2
	}{
		2\\
		0
	}
	\end{examplebox}
	
	\subsection*{说明/提示}
	对于区间$[1,3]$,$val_{l,r}=114514^2\times10$,可以被$114514^2$整除但是不能被$114514^3$整除,故$k$为2。
	
	对于区间$[2,2]$,$val_{l,r}=10$,可以被$114514^0$整除但是不能被$114514^1$整除,故$k$为0。
	
\end{problem}}{}
	\IfFileExists{problems/E.tex}{\begin{problem}[古希腊掌管三角形的神]
	
	秧歌 (又被称为年轻的云) 是古希腊掌管三角形的神,无所畏惧的你来到秧歌的面前想要挑战他的权威,秧歌不屑一顾地对你说:“何意味?正确回答这个问题你才有资格触碰神圣的三角形”。
	
	秧歌将 $n$ 根木棍摆在你的面前,第 $i$ 根木棍的长度为 $a_i$。接着他会提出 $q$ 个对木棍的查询,每个查询会给出两个整数 $l,r$ ( $1\leq l< r\leq n$,且 $r - l + 1 \geq 6$ )。你需要回答,能否从第 $l$ 到 $r$ (包含$l,r$)个木棍中,挑选出 $6$ 根不同的木棍,组成 $2$ 个非退化三角形(若一个三角形的三边长 $a、b、c$ 满足 $a < b + c$且$b < a + c$且$c < a + b$,则称其为非退化三角形)。
	
	
	
	\subsection*{输入描述}
	第一行两个整数 $n$ 和 $q$ ( $6\leq n\leq 10^5,1\leq q\leq 10^5$)分别表示木棍的数量和查询的数量。
	
	第二行 $n$ 个整数 $a_1, a_2......a_n$ ( $1\leq a_i\leq 10^9$ )表示木棍长度。
	
	接下来 $q$ 行,每行两个整数 $l,r$ ( $1\leq l < r\leq n$,且 $r - l + 1 \geq 6$ )表示询问。
	
	\subsection*{输出描述}
	对于每个查询,若能组成 $2$ 个非退化三角形,输出一行 YES,否则输出一行 NO。
	
	\subsection*{样例}
	\begin{examplebox}{
		10 5\\
		5 2 2 10 4 10 6 1 5 3\\
		1 6\\
		2 7\\
		2 8\\
		5 10\\
		4 10
	}{
		YES\\
		NO\\
		YES\\
		NO\\
		YES
	}
	\end{examplebox}
	
	
	\subsection*{说明/提示}
	第一个查询中,木棍长度为 $[5, 2, 2, 10, 4, 10]$,可选择两组木棍 $[2, 4, 5]$ 和 $[2, 10, 10]$,分别组成 $2$ 个非退化三角形。
	
	第二个查询中,木棍长度为 $[2, 2, 10, 4, 10, 6]$,无法组成 $2$ 个非退化三角形。
	
	第三个查询中,木棍长度为 $[2, 2, 10, 4, 10, 6, 1]$,可选择两组木棍 $[1, 2, 2]$ 和 $[4, 10, 10]$,分别组成 $2$ 个非退化三角形。
	
	第四个查询中,木棍长度为 $[4, 10, 6, 1, 5, 3]$,无法组成 $2$ 个非退化三角形。
	
	第五个查询中,木棍长度为 $[10, 4, 10, 6, 1, 5, 3]$,可选择两组木棍 $[1, 10, 10]$ 和 $[3, 4, 5]$,分别组成 $2$ 个非退化三角形。
\end{problem}}{}
	\IfFileExists{problems/F.tex}{\begin{problem}[水+水=水]
	
	在一个二维网格世界中,水超现实地流动着。网格大小为$m \times n$,左上角为 $(0,0)$,右下角为$(m,n)$。
	
	\begin{quote}
		「水源」:水流动的中心。
		
		「流动」:水会以水源为中心向周围流动,每流动一格,流量-1。流量降至0时,水不再继续流动。
		
		「流量」:一个格子可能会被多处水源流动到,该格子的流量为所有水源流动到此处的流量的最大值。水源的流量为5。
		
		「水的生成」:当一个格子周围四个存在至少两个水源,则该格子变为水源。水源会不断生成,直到没有格子满足生成水源的条件。
	\end{quote}
	
	例如,水流的最终状态可能是这样:
	
	\begin{figure}[H]
		\centering
		\includegraphics[width=0.5\textwidth]{flow.png}
		\caption{水的流动} % 图片标题
		\label{flow} % 标签,用于交叉引用
	\end{figure}
	
	如图\ref{to_merge}所示,被标记的格子周围存在两个水源,因此会形成新的水源。
	\begin{figure}[H]
		\centering
		\includegraphics[width=0.5\textwidth]{to_merge.png}
		\caption{中间状态} % 图片标题
		\label{to_merge} % 标签,用于交叉引用
	\end{figure}
	
	新的水源生成后,最终状态如图\ref{merged}所示。
	\begin{figure}[H]
		\centering
		\includegraphics[width=0.5\textwidth]{merged.png}
		\caption{新的水源生成} % 图片标题
		\label{merged} % 标签,用于交叉引用
	\end{figure}
	
	世界创立之初,k个格子上存在水源,随之,水开始流动、生成。请你计算最终每个格子上的流量。
	
	\subsection*{输入描述}
	第一行一个整数$T(1\leq T\leq 100)$表示数据组数,接下来按组给出数据。
	
	对于每组数据:
	\begin{quote}
		第一行包含三个整数 $m, n, k$($1 \le m, n \le 20$),表示网格的行数和列数、初始水源数量。
		
		接下来 $k$ 行,每行包含两个整数 $x_i, y_i$($0 \le x_i < m,\ 0 \le y_i < n$)表示初始水源的坐标。保证坐标不会重复。
	\end{quote}
	
	\subsection*{输出描述}
	对于每组数据,输出m行,每行包含n个整数,表示最终每个格子的流量。
	
	\subsection*{样例}
	\begin{examplebox}{
		1\\
		4 6 2\\
		0 0\\
		1 1
	}{
		554321\\
		554321\\
		443210\\
		332100
	}
	\end{examplebox}
	
	\subsection*{说明/提示}
	在$(0,1)$和$(1,0)$处生成了新的水源。
	
\end{problem}}{}
	\IfFileExists{problems/G.tex}{\begin{problem}[恋爱物语]
	
	广袤的互联网中有这样一个QQ群,其中不少人向往甜甜的恋爱。
	
	\begin{quote}
		令:想要甜甜的恋爱怎么办(2025-11-04 09:49)
		
		涵:帮我算算我什么时候可以谈恋爱(2025-11-04 20:52)
		
		werwer:想谈甜甜的恋爱了(2025-11-03 18:23)
		
		明末美食家:我也想了解一下恋爱的细节(2025-10-29 23:03)
	\end{quote}
	
	QQ群中有$n$个男生和$m$个女生,每个人有且仅有一个喜欢的异性。如果两个人互相喜欢,他们会成为一对情侣。
	
	群主小M想知道QQ群中现在有多少对互相喜欢的关系。
	
	\subsection*{输入描述}
	
	第一行一个整数 $T(1\leq T\leq 100)$ 表示数据组数,接下来按组给出数据。
	
	对于每组数据:
	
	\begin{quote}
		第一行两个整数 $n,m(2\leq n,m\leq100)$ 代表男生、女生人数。
		
		第二行 $n$ 个整数 $a_1,a_2......a_n(1\leq a_i\leq m)$,第 $i$ 个男生喜欢第 $a_i$ 个女生。
		
		第三行 $m$ 个整数 $b_1,b_2......b_m(1\leq b_i\leq n)$,第 $i$ 个女生喜欢第 $b_i$ 个男生。
	\end{quote}
	
	\subsection*{输出描述}
	
	对于每组数据,输出一行一个整数,表示相互喜欢关系的数量。
	
	\subsection*{样例}
	\begin{examplebox}{
		2\\
		4 5\\
		1 1 3 3\\
		1 2 4 3 1\\
		5 4\\
		1 2 3 4 4\\
		3 2 2 1
	}{
		2\\
		1
	}
	\end{examplebox}
	
	\subsection*{说明/提示}
	第一组数据,互相喜欢的关系有$(1,1),(4,3)$。
	
	第二组数据,互相喜欢的关系有$(2,2)$。
	
\end{problem}}{}
	\IfFileExists{problems/H.tex}{\begin{problem}[拼图]
	
	小C有一个特别的拼图,拼图有以下四种组件:
	
	\begin{figure}[H]
		\centering
		\includegraphics[width=0.75\textwidth]{puzzle.png}
		\caption{四种组件} % 图片标题
		\label{component} % 标签,用于交叉引用
	\end{figure}
	
	每个组件的左右两侧各有一个连接结构:要么是凸起,要么是凹陷。\textbf{组件不可旋转}。
	
	两个组件能够连接的条件是:左侧组件的右侧连接结构与右侧组件的左侧连接结构互为相反类型 (即凸起与凹陷配对)。
	
	拼图中每种类型的组件数量分别为 $c_1, c_2, c_3, c_4$ (数量按照图片中的顺序给出)。若能将所有组件拼接成一条完整的长链,则认为拼图完成。
	
	你需要帮助小C求出有多少种不同的拼接方式。
	
	\subsection*{输入描述}
	第一行一个整数 $T (1 \leq T \leq 2\times 10^5 ) $表示数据的组数。
	
	接下来$T$行,每行四个整数 $ c_1,c_2,c_3,c_4 (0 \leq c_i \leq 10^6)$表示每种组件的数量。
	
	保证所有测试用例的 $c_i$ 之和不超过 $4 \times 10^6$。
	
	\subsection*{输出描述}
	对于每组数据,输出一行一个整数,表示完成拼图的方案数。
	
	由于答案可能非常大,所以答案对 $998244353$ 取模。
	
	如果无法完成这个拼图,输出$0$。如果$c_1=c_2=c_3=c_4=0$,方案数为$1$。
	
	\subsection*{样例}
	\begin{examplebox}{
		5\\
		0 0 0 4\\
		1 1 1 1\\
		5 5 0 2\\
		4 6 8 10\\
		900000 900000 900000 900000
	}{
		1\\
		4\\
		36\\
		0\\
		794100779
	}
	\end{examplebox}
	
	\subsection*{说明/提示}
	如果存在$i$使得两种方案的第$i$个组件不同,那么这两种方法是不同的。
	
\end{problem}}{}
	\IfFileExists{problems/I.tex}{\begin{problem}[像素水果忍者]
	
	年轻的云(又被称为秧歌)最近迷上了一款叫作《水果忍者》的游戏。在这个游戏中,屏幕上会出现水果,你需要在屏幕上滑动,如果滑动的路径和水果有重合的部分,你就能切开这个水果并获得分数。
	
	由于这个游戏只能免费玩114514秒,秧歌决定自己开发一款新游戏《像素水果忍者》。在这里,水果都是矩形的且不会倾斜,而你画出路径都是线段。秧歌想让你帮他完成判断水果是否被切开的功能。
	
	游戏可以抽象为一个平面直角坐标系。每次,给出一条线段的左端点$(x_1, y_1)$ 和右端点 $(x_2, y_2)$,如果线段垂直于$x$轴则按任意顺序给出;给出一个矩形的左下角 $(x_l, y_l)$ 和右上角 $(x_r, y_r)$ ,保证四边平行于坐标轴;判断线段和矩形有没有交集。
	
	\subsection*{输入描述}
	第一行一个整数 $T(1\leq T\leq 10^5)$表示数据组数。接下来按组给出数据。
	
	对于每组数据:
	\begin{quote}
	第一行四个整数 $x_1,y_1,x_2,y_2$表示线段的左右端点。
	
	第二行四个整数 $x_l,y_l,x_r,y_r$表示矩形左下角和右上角。
	\end{quote}
	
	保证给出的坐标值的绝对值不超过 $10^4$。保证线段不会退化成点,矩形不会退化成线段或点。
	
	\subsection*{输出描述}
	对于每组数据,如果有交集则输出一行YES,否则输出一行NO。
	
	\subsection*{样例}
	\begin{examplebox}{
		3\\
		0 0 5 5\\
		0 0 10 10\\
		0 0 5 5\\
		6 6 7 7\\
		0 0 5 5\\
		1 1 2 2
	}{
		YES\\
		NO\\
		YES
	}
	\end{examplebox}
	
	\subsection*{说明/提示}
	如果存在一个点既在线段上(含端点)也在矩形中(含边界),则认为线段和矩形有交集。
\end{problem}}{}
	\IfFileExists{problems/J.tex}{\begin{problem}[我不是药神]
	Alice 和 Bob 开了一家药店。这天他们进货了 $n$ 瓶药,每瓶药中有 $n$ 粒药丸。Alice 在检查这批药的时候,发现这 $n$ 瓶药中有且仅有一瓶是假药,同时她发现,如果一瓶药是真药,则里面的每粒药丸都重 500mg;如果一瓶药是假药,则里面的每粒药丸都重 510mg。Alice 告诉 Bob 有一瓶是假药,也告诉了 Bob 真药和假药的区别。她想考考 Bob,于是规定了一种称重方式:
	
	\begin{quote}
		从每瓶药中取出任意粒药丸,一并称出取出的药丸的总重量。
		
		形式化地,你可以任意构造一个数组$a_1,a_2......a_n(0\leq a_i\leq n)$,从第$i$瓶中取出$a_i$粒药丸并消耗一次称重次数得到这 $\sum\limits_{i = 1}^{k}a_i$ 粒药丸的总重量。
	\end{quote}
	
	Alice问Bob,在她给定的称重方式下,至少要称重多少次才能确定第几瓶药是假的。但是 Bob 一时想不出来,于是找了你来帮忙。
	
	\subsection*{输入描述}
	第一行一个整数 $T(1 \leq T \leq 10^5)$表示数据组数。
	
	随后$T$行,每行一个整数 $n(1 \leq n \leq 10^9)$,表示药的瓶数 。
	
	\subsection*{输出描述}
	对于每组数据,输出一行一个整数表示最少称重次数。
	
	\subsection*{样例}
	\begin{examplebox}{
		2\\
		1\\
		2
	}{
		0\\
		1
	}
	\end{examplebox}
	
	\subsection*{说明/提示}
	
	如果有两瓶药,我们只从某一瓶中取一粒药丸并称重。若该药丸重510mg,则对应的那瓶药为假;否则,另一瓶为假。称重一次即可,答案为1。
	
\end{problem}}{}
	\IfFileExists{problems/K.tex}{\begin{problem}[上进的冰妖精]
	
	小\circledtext{9}是一只冰妖精。她十分上进,讨厌一切下降的东西。
	这天,她来到雾之湖边玩耍,发现了一排冰柱,第$i$根冰柱的高度为$h_i$。但是,这些冰柱参差不齐,也就是,其中可能含有「下降」!
	
	小\circledtext{9}认为,如果存在$i(1\leq i < n)$使得$h_i>h_{i+1}$,那么这些冰柱是不上进的。否则,这些冰柱是上进的。
	作为冰妖精,小\circledtext{9}有操纵冰程度的能力。每次,她可以选择一些冰柱,并且让它们的高度-1。
	
	最少需要使用多少次能力才能让这些冰柱变得上进呢?作为大数学家的小\circledtext{9}并不屑于计算这么简单的问题,于是把这个任务交给了你。
	
	\subsection*{输入描述}
	第一行一个整数 $n(1\le n\le 10^6)$ 表示序列元素的个数。
	
	第二行 $n(1\le h_i\le10^9)$ 个整数,表示冰柱的高度。
	
	\subsection*{输出描述}
	输出一行一个整数,表示最少的使用能力的次数。
	
	\subsection*{样例}
	\begin{examplebox}{
		3\\
		1 3 2
	}{
		1
	}
	\end{examplebox}
	
	
	\subsection*{说明/提示}
	
	选择第二根柱子并使其高度-1,冰柱的高度变为[1,2,2],满足条件。故使用一次能力即可,答案为1。
	
\end{problem}}{}
	\IfFileExists{problems/L.tex}{\begin{problem}[《世界》]
	
	小M喜欢玩一款叫作《世界》的游戏。
	
	\subsubsection*{构成与法则}
	《世界》是一个由方块和地面组成的三维空间,其中,地面是一种由特殊的同种方块构成的无限大平面,甚至可以延申到《世界》外部;方块一共有三种,分别是沙子、石头和火把。
	我们可以将《世界》的大小表示为 $X,Y,Z$,将内部的位置集合表示为 
	\[\{(x,y,z)|{x\in[0,X-1],y\in[0,Y-1],z\in[0,Z-1]}\}\]
	
	将地面表示为\[\{(x,y,z)|z=-1\}\]
	
	《世界》中有一套特殊的物理法则:
	\begin{quote}
		方块可以被放置某位置,当且仅当该位置在《世界》内部,且该位置当前没有方块,且相邻六个面的位置已存在方块或为地面。
		
		当方块的下方没有紧贴着地面或者方块,我们认为这个方块是悬浮的。石头可以悬浮而不会落下;沙子悬浮会下落。
		
		火把是一种特殊的方块,只能附着在相邻的非火把的方块(包括地面)的侧面或上面,且附着关系一旦建立就不会改变。当火把附着的方块被破坏/下落,该火把立刻消失;其余情况下火把不会改变状态。
	\end{quote}
	
	显然,当没有沙子悬浮的时候,《世界》是稳定的。
	
	\subsubsection*{你的任务}
	小M在心中构想了一份计划。计划分为$N$条指令,每一条可能是放置或者破坏。他想让你检测计划的可行性,因此,你需要按顺序检测每条指令是否合法。
	
	如果当前指令合法,执行之,输出一行GOODJOB,等待《世界》稳定后再进行下一步;否则,跳过当前指令,输出一行 AREYOUKIDDINGME。所有指令检测完毕后,输出《世界》的状态,具体输出方法将在下文说明。
	
	\subsubsection*{指令种类}
	
	\textbf{放石头/沙子:}输入格式为一个字符串和三个整数:$PUT\_STONE\ x\ y\ z$或$PUT\_SAND\ x\ y\ z$,参数表示放置位置。
	如果放置位置没有方块且相邻六个面的位置存在方块或为地面,则合法。
	
	\textbf{插火把:}输入格式为一个字符串和四个整数,$PUT\_TORCH\ x\ y\ z\ f$,参数表示放置位置和附着的方向。如果放置位置没有方块且附着位置上存在不为火把的方块或为地面,则合法。$f$ 有以下几种取值:
	
	\begin{quote}
		1:附着在坐标为 $(x+1,y,z)$ 上;
		
		2:附着在坐标为 $(x-1,y,z)$ 上;
		
		3:附着在坐标为 $(x,y+1,z)$ 上;
		
		4:附着在坐标为 $(x,y-1,z)$ 上;
		
		5:附着在坐标为 $(x,y,z-1)$ 上。
	\end{quote}
	
	\textbf{破坏:}输入格式为一个字符串和三个整数,$DESTROY\ x\ y\ z$ 参数表示破坏位置。如果破坏位置上存在方块,则合法。
	
	\subsubsection*{状态输出}
	
	你需要按照下方伪代码表示的方式输出《世界》状态:
	
	\begin{pseudocode}
		OUTPUT "THE WORLD"
		NEWLINE
		FOR k FROM 0 TO Z-1
		\indentcode FOR i FROM 0 TO X-1
		\indentcode\indentcode FOR j FROM 0 TO Y-1
		\indentcode\indentcode\indentcode IF stone on (i,j,k) THEN
		\indentcode\indentcode\indentcode\indentcode OUTPUT 'r'
		\indentcode\indentcode\indentcode ElSE IF sand on (i,j,k) THEN
		\indentcode\indentcode\indentcode\indentcode OUTPUT 's'
		\indentcode\indentcode\indentcode ELSE IF torch on (i,j,k) THEN
		\indentcode\indentcode\indentcode\indentcode IF f IS 1 THEN
		\indentcode\indentcode\indentcode\indentcode\indentcode OUTPUT 'v'(小写字母v)
		\indentcode\indentcode\indentcode\indentcode ELSE IF f IS 2 THEN
		\indentcode\indentcode\indentcode\indentcode\indentcode OUTPUT '\textasciicircum'(shift+6)
		\indentcode\indentcode\indentcode\indentcode ELSE IF f IS 3 THEN
		\indentcode\indentcode\indentcode\indentcode\indentcode OUTPUT '>'(shift+句号)
		\indentcode\indentcode\indentcode\indentcode ELSE IF f IS 4 THEN
		\indentcode\indentcode\indentcode\indentcode\indentcode OUTPUT '<'(shift+逗号)
		\indentcode\indentcode\indentcode\indentcode ELSE
		\indentcode\indentcode\indentcode\indentcode\indentcode OUTPUT 't'
		\indentcode\indentcode\indentcode\indentcode ENDIF
		\indentcode\indentcode\indentcode ELSE
		\indentcode\indentcode\indentcode\indentcode OUTPUT 'a'
		\indentcode\indentcode\indentcode ENDIF
		\indentcode\indentcode ENDFOR
		\indentcode\indentcode NEWLINE
		\indentcode ENDFOR
		\indentcode NEWLINE
		ENDFOR
	\end{pseudocode}
	
	\subsection*{输入描述}
	
	第一行三个整数 $X,Y,Z(1\leq X,Y,Z\leq10)$ 表示《世界》的大小。
	
	第二行一个整数 $N(1\leq N\leq 10^5)$ 代表指令数量。
	
	随后 $N$ 行,每行一条指令,具体格式见题目描述。
	
	对于所有指令,保证$0\leq x<X,0\leq y<Y,0\leq z<Z$。
	
	\subsection*{输出描述}
	
	对于每条指令,输出一行GOODJOB或仅AREYOUKIDDINGME。
	
	最后输出《世界》的状态。
	
	\subsection*{样例}
	\begin{examplebox}{
		2 2 2\\
		6\\
		PUT\_SAND 0 0 1\\
		PUT\_STONE 1 0 0\\
		PUT\_TORCH 0 0 0 1\\
		PUT\_SAND 0 0 1\\
		DESTROY 0 1 0\\
		DESTROY 1 0 0
	}{
		AREYOUKIDDINGME\\
		GOODJOB\\
		GOODJOB\\
		GOODJOB\\
		AREYOUKIDDINGME\\
		GOODJOB\\
		sa\\
		aa\\
		\\
		aa\\
		aa
	}
	\end{examplebox}
	
	\subsection*{说明/提示}
	样例演示了:放置石头,在石头侧面附着火把,在火把上放沙子;破坏石头,火把消失,沙子下落。
	\begin{figure}[H]
		\centering
		\includegraphics[width=1\textwidth]{destroy.png}
		\caption{小M的涂鸦} % 图片标题
		\label{destroy} % 标签,用于交叉引用
	\end{figure}
	
\end{problem}}{}
	\IfFileExists{problems/M.tex}{\begin{problem}[晚餐]
	
	小M想吃丰盛的晚餐,他决定出门购买一些食材,但是他想尽可能少花力气。
	
	小M的家和店铺之间的连通关系可以抽象为一颗有$n$个结点的树,根结点代表小M的家,其他结点代表店铺。
	每家店铺最多出售一种小M需要的食材,且每种需要的食材最多被一家店铺出售。
	小M可以在边上行走,耗费的力气值为(携带物品重量 $\times$ 边长)。
	
	小M从家开始在树上移动,经过店铺时,他可以选择在该店铺买下任意重量的食材。在此后的移动中,他必须携带这些食材;但是当经过家时,他可以将这些食材卸下,并重新回到空手的状态。
	
	请求出小M购买完所有需要的食材并放回家最少需要花费的力气值。
	
	\subsection*{输入描述}
	
	第一行一个整数 $T(1\leq T\leq 10^5)$ 表示数据组数,接下来按组给出数据。
	
	对于每组数据:
	
	\begin{quote}
		第一行两个整数 $n(1\leq n\leq 10^5),r$ 表示树的结点个数、根结点编号(结点编号从1开始)。
		
		随后 $n-1$ 行,每行两个整数 $u,v,w(1\leq w\leq 1000)$ ,代表编号为 $u,v,w$ 的结点之间有一条长度为 $w$ 边。
		
		随后一行 $n$ 个整数 $a_1,a_2......a_n(1\leq v_i \leq 1000)$,$a_i$ 代表编号为 $i$ 的结点代表的店铺中需要购买的食材的重量。$a_r=0$ 。
	\end{quote}
	
	数据保证 $\sum n\leq10^6$
	
	\subsection*{输出描述}
	
	对于每组数据,输出一行,包含一个整数,代表最少需要花费的力气值。
	
	\subsection*{样例}
	\begin{examplebox}{
		2\\
		2 1\\
		1 2 1\\
		0 1\\
		4 1\\
		1 2 2\\
		2 3 1\\
		3 4 1\\
		0 1 1 1
	}{
		1\\
		9
	}
	\end{examplebox}
	
	\subsection*{说明/提示}
	第二组数据的树结构如图所示:
	
	\begin{figure}[H]
		\centering
		\includegraphics[width=0.5\textwidth]{graph.png}
		\caption{树} % 图片标题
		\label{graph} % 标签,用于交叉引用
	\end{figure}
	
	小M可以从家里出发,空手走到结点4。
	
	在结点4,购买全部的食材。携带物品重量变为1。
	
	向上走到结点3,花费力气值为$1\times 1=1$。
	
	在结点3,购买全部的食材。携带物品重量变为2。
	
	向上走到结点2,花费力气值为$2\times 1=2$。
	
	在结点2,购买全部的食材。携带物品重量变为3。
	
	向上走到结点1,花费力气值为$3\times 2=6$。
	
	放下所有食材。
	
	总共花费的力气值为 1+2+6=9。
	
\end{problem}}{}
	
	
\end{document}
