\begin{problem}[《世界》]
	
	小M喜欢玩一款叫作《世界》的游戏。
	
	\subsubsection*{构成与法则}
	《世界》是一个由方块和地面组成的三维空间,其中,地面是一种由特殊的同种方块构成的无限大平面,甚至可以延申到《世界》外部;方块一共有三种,分别是沙子、石头和火把。
	我们可以将《世界》的大小表示为 $X,Y,Z$,将内部的位置集合表示为 
	\[\{(x,y,z)|{x\in[0,X-1],y\in[0,Y-1],z\in[0,Z-1]}\}\]
	
	将地面表示为\[\{(x,y,z)|z=-1\}\]
	
	《世界》中有一套特殊的物理法则:
	\begin{quote}
		方块可以被放置某位置,当且仅当该位置在《世界》内部,且该位置当前没有方块,且相邻六个面的位置已存在方块或为地面。
		
		当方块的下方没有紧贴着地面或者方块,我们认为这个方块是悬浮的。石头可以悬浮而不会落下;沙子悬浮会下落。
		
		火把是一种特殊的方块,只能附着在相邻的非火把的方块(包括地面)的侧面或上面,且附着关系一旦建立就不会改变。当火把附着的方块被破坏/下落,该火把立刻消失;其余情况下火把不会改变状态。
	\end{quote}
	
	显然,当没有沙子悬浮的时候,《世界》是稳定的。
	
	\subsubsection*{你的任务}
	小M在心中构想了一份计划。计划分为$N$条指令,每一条可能是放置或者破坏。他想让你检测计划的可行性,因此,你需要按顺序检测每条指令是否合法。
	
	如果当前指令合法,执行之,输出一行GOODJOB,等待《世界》稳定后再进行下一步;否则,跳过当前指令,输出一行 AREYOUKIDDINGME。所有指令检测完毕后,输出《世界》的状态,具体输出方法将在下文说明。
	
	\subsubsection*{指令种类}
	
	\textbf{放石头/沙子:}输入格式为一个字符串和三个整数:$PUT\_STONE\ x\ y\ z$或$PUT\_SAND\ x\ y\ z$,参数表示放置位置。
	如果放置位置没有方块且相邻六个面的位置存在方块或为地面,则合法。
	
	\textbf{插火把:}输入格式为一个字符串和四个整数,$PUT\_TORCH\ x\ y\ z\ f$,参数表示放置位置和附着的方向。如果放置位置没有方块且附着位置上存在不为火把的方块或为地面,则合法。$f$ 有以下几种取值:
	
	\begin{quote}
		1:附着在坐标为 $(x+1,y,z)$ 上;
		
		2:附着在坐标为 $(x-1,y,z)$ 上;
		
		3:附着在坐标为 $(x,y+1,z)$ 上;
		
		4:附着在坐标为 $(x,y-1,z)$ 上;
		
		5:附着在坐标为 $(x,y,z-1)$ 上。
	\end{quote}
	
	\textbf{破坏:}输入格式为一个字符串和三个整数,$DESTROY\ x\ y\ z$ 参数表示破坏位置。如果破坏位置上存在方块,则合法。
	
	\subsubsection*{状态输出}
	
	你需要按照下方伪代码表示的方式输出《世界》状态:
	
	\begin{pseudocode}
		OUTPUT "THE WORLD"
		NEWLINE
		FOR k FROM 0 TO Z-1
		\indentcode FOR i FROM 0 TO X-1
		\indentcode\indentcode FOR j FROM 0 TO Y-1
		\indentcode\indentcode\indentcode IF stone on (i,j,k) THEN
		\indentcode\indentcode\indentcode\indentcode OUTPUT 'r'
		\indentcode\indentcode\indentcode ElSE IF sand on (i,j,k) THEN
		\indentcode\indentcode\indentcode\indentcode OUTPUT 's'
		\indentcode\indentcode\indentcode ELSE IF torch on (i,j,k) THEN
		\indentcode\indentcode\indentcode\indentcode IF f IS 1 THEN
		\indentcode\indentcode\indentcode\indentcode\indentcode OUTPUT 'v'(小写字母v)
		\indentcode\indentcode\indentcode\indentcode ELSE IF f IS 2 THEN
		\indentcode\indentcode\indentcode\indentcode\indentcode OUTPUT '\textasciicircum'(shift+6)
		\indentcode\indentcode\indentcode\indentcode ELSE IF f IS 3 THEN
		\indentcode\indentcode\indentcode\indentcode\indentcode OUTPUT '>'(shift+句号)
		\indentcode\indentcode\indentcode\indentcode ELSE IF f IS 4 THEN
		\indentcode\indentcode\indentcode\indentcode\indentcode OUTPUT '<'(shift+逗号)
		\indentcode\indentcode\indentcode\indentcode ELSE
		\indentcode\indentcode\indentcode\indentcode\indentcode OUTPUT 't'
		\indentcode\indentcode\indentcode\indentcode ENDIF
		\indentcode\indentcode\indentcode ELSE
		\indentcode\indentcode\indentcode\indentcode OUTPUT 'a'
		\indentcode\indentcode\indentcode ENDIF
		\indentcode\indentcode ENDFOR
		\indentcode\indentcode NEWLINE
		\indentcode ENDFOR
		\indentcode NEWLINE
		ENDFOR
	\end{pseudocode}
	
	\subsection*{输入描述}
	
	第一行三个整数 $X,Y,Z(1\leq X,Y,Z\leq10)$ 表示《世界》的大小。
	
	第二行一个整数 $N(1\leq N\leq 10^5)$ 代表指令数量。
	
	随后 $N$ 行,每行一条指令,具体格式见题目描述。
	
	对于所有指令,保证$0\leq x<X,0\leq y<Y,0\leq z<Z$。
	
	\subsection*{输出描述}
	
	对于每条指令,输出一行GOODJOB或仅AREYOUKIDDINGME。
	
	最后输出《世界》的状态。
	
	\subsection*{样例}
	\begin{examplebox}{
		2 2 2\\
		6\\
		PUT\_SAND 0 0 1\\
		PUT\_STONE 1 0 0\\
		PUT\_TORCH 0 0 0 1\\
		PUT\_SAND 0 0 1\\
		DESTROY 0 1 0\\
		DESTROY 1 0 0
	}{
		AREYOUKIDDINGME\\
		GOODJOB\\
		GOODJOB\\
		GOODJOB\\
		AREYOUKIDDINGME\\
		GOODJOB\\
		sa\\
		aa\\
		\\
		aa\\
		aa
	}
	\end{examplebox}
	
	\subsection*{说明/提示}
	样例演示了:放置石头,在石头侧面附着火把,在火把上放沙子;破坏石头,火把消失,沙子下落。
	\begin{figure}[H]
		\centering
		\includegraphics[width=1\textwidth]{destroy.png}
		\caption{小M的涂鸦} % 图片标题
		\label{destroy} % 标签,用于交叉引用
	\end{figure}
	
\end{problem}