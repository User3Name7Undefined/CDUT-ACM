\begin{problem}[拼手速]
	
	6111年52月02日, 小M参加了第114514届程协杯。
	这次程协杯持续时间为 $10^{10^{10}}$ 分钟(以下默认时间单位为分钟),有 $n$ 道题目,采用CIALLO赛制。
	
	CIALLO赛制和大家熟知的ACM赛制相似,不同的是,CIALLO赛制没有罚时!
	每道题目的用时为比赛开始到该题目被正确解决的时间,最终,选手的总用时是所有被正确解决的题目的用时的和。
	
	每天刷水题的小M瞬间看完了所有的题目,并且算出了每道题目要花费的时间。他立刻开始解题,每道题目解出后立刻开始解下一题,保证不会出错。
	
	小M可以以任意顺序解题,请求出小M的总用时最低是多少。
	
	\subsection*{输入描述}
	第一行一个整数 $T(1\leq T\leq 100)$ 表示数据组数,接下来按组给出数据。
	
	对于每组数据:
	
	\begin{quote}
		第一行一个整数 $n(1\leq n\leq20)$ 代表程协杯的题目数量;
		
		随后一行 $n$ 个整数 $t_1,t_2......t_n(0\leq t_i\leq 1000)$,$t_i$代表道第$i$道题目花费的时间。
	\end{quote}
	
	\subsection*{输出描述}
	对于每组数据,输出一行一个整数,代表小M最低的总用时。
	
	\subsection*{样例}
	\begin{examplebox}{
		2\\
		2\\
		2 1\\
		5\\
		1 1 1 1 1
	}{
		4\\
		15
	}
	\end{examplebox}
	
	\subsection*{说明/提示}
	对于第一组数据:最优的解题顺序是$[2,1]$,用时为$[1,3]$,总用时为4。
	
	对于第二组数据:解题顺序是$[1,2,3,4,5]$,用时为$[1,2,3,4,5]$,总用时为15。
	
	\textbf{显然,选手应当先解决花费时间少的题目。}
	
\end{problem}