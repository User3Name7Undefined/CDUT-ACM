\begin{problem}[对称]
	埃及法老Aiji Falao是一位暴君,他喜欢抓人来解决自己出的谜题,成功解决能获得一定的赏金,但如果无法解决就会被当作祭品。
	很不幸,这天,你被抓住了;但是很幸运,你的大脑连接到了第114154届程协杯比赛现场的某台电脑上。
	
	Aiji Falao向你发起了询问:
	
	\begin{quote}
		我有一个$n\times n$的棋盘和$n$个相同的棋子。每一行、每一列最多放一个棋子,而且棋子摆放的形状必须关于主对角线(从左上到右下)对称。告诉我,棋子有多少种摆放的方式?
	\end{quote}
	
	为了避免成为祭品,你需要立即编程计算出有多少种摆法。
	
	\subsection*{输入描述}
	第一行一个整数 $T(1\le T\le 10^6)$ 表示数据组数。
	
	接下来 $T$ 行,每行一个整数 $n(1\le n\le 10^6)$ 表示棋盘大小和土块个数。
	
	\subsection*{输出描述}
	对于每组数据,输出一行一个整数,表示摆法的数量。
	
	由于答案可能非常大,所以答案对 $10^9+7$ 取模。
	
	\subsection*{样例}
	\begin{examplebox}{
		3\\
		1\\
		2\\
		3
	}{
		1\\
		2\\
		4
	}
	\end{examplebox}
	
	\subsection*{说明/提示}
	如果存在位置$(i,j)$,使得两种摆法在$(i,j)$上一个有棋子而另一个没有,那么这两种摆法是不同的。
	
	当$n=1$时,摆法有$(1,1)$。
	
	当$n=2$时,摆法有$(1,1)(2,2)$、$(1,2)(2,1)$。
	
	当$n=3$时,摆法有$(1,1)(2,2)(3,3)$、$(1,1)(3,2)(2,3)$、$(1,3)(2,2)(3,1)$、$(1,2)(2,1)(3,3)$。
\end{problem}