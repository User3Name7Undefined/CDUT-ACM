\begin{problem}[晚餐]
	
	小M想吃丰盛的晚餐,他决定出门购买一些食材,但是他想尽可能少花力气。
	
	小M的家和店铺之间的连通关系可以抽象为一颗有$n$个结点的树,根结点代表小M的家,其他结点代表店铺。
	每家店铺最多出售一种小M需要的食材,且每种需要的食材最多被一家店铺出售。
	小M可以在边上行走,耗费的力气值为(携带物品重量 $\times$ 边长)。
	
	小M从家开始在树上移动,经过店铺时,他可以选择在该店铺买下任意重量的食材。在此后的移动中,他必须携带这些食材;但是当经过家时,他可以将这些食材卸下,并重新回到空手的状态。
	
	请求出小M购买完所有需要的食材并放回家最少需要花费的力气值。
	
	\subsection*{输入描述}
	
	第一行一个整数 $T(1\leq T\leq 10^5)$ 表示数据组数,接下来按组给出数据。
	
	对于每组数据:
	
	\begin{quote}
		第一行两个整数 $n(1\leq n\leq 10^5),r$ 表示树的结点个数、根结点编号(结点编号从1开始)。
		
		随后 $n-1$ 行,每行两个整数 $u,v,w(1\leq w\leq 1000)$ ,代表编号为 $u,v,w$ 的结点之间有一条长度为 $w$ 边。
		
		随后一行 $n$ 个整数 $a_1,a_2......a_n(1\leq v_i \leq 1000)$,$a_i$ 代表编号为 $i$ 的结点代表的店铺中需要购买的食材的重量。$a_r=0$ 。
	\end{quote}
	
	数据保证 $\sum n\leq10^6$
	
	\subsection*{输出描述}
	
	对于每组数据,输出一行,包含一个整数,代表最少需要花费的力气值。
	
	\subsection*{样例}
	\begin{examplebox}{
		2\\
		2 1\\
		1 2 1\\
		0 1\\
		4 1\\
		1 2 2\\
		2 3 1\\
		3 4 1\\
		0 1 1 1
	}{
		1\\
		9
	}
	\end{examplebox}
	
	\subsection*{说明/提示}
	第二组数据的树结构如图所示:
	
	\begin{figure}[H]
		\centering
		\includegraphics[width=0.5\textwidth]{graph.png}
		\caption{树} % 图片标题
		\label{graph} % 标签,用于交叉引用
	\end{figure}
	
	小M可以从家里出发,空手走到结点4。
	
	在结点4,购买全部的食材。携带物品重量变为1。
	
	向上走到结点3,花费力气值为$1\times 1=1$。
	
	在结点3,购买全部的食材。携带物品重量变为2。
	
	向上走到结点2,花费力气值为$2\times 1=2$。
	
	在结点2,购买全部的食材。携带物品重量变为3。
	
	向上走到结点1,花费力气值为$3\times 2=6$。
	
	放下所有食材。
	
	总共花费的力气值为 1+2+6=9。
	
\end{problem}