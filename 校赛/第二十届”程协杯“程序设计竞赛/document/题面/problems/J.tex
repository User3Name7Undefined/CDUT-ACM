\begin{problem}[我不是药神]
	Alice 和 Bob 开了一家药店。这天他们进货了 $n$ 瓶药,每瓶药中有 $n$ 粒药丸。Alice 在检查这批药的时候,发现这 $n$ 瓶药中有且仅有一瓶是假药,同时她发现,如果一瓶药是真药,则里面的每粒药丸都重 500mg;如果一瓶药是假药,则里面的每粒药丸都重 510mg。Alice 告诉 Bob 有一瓶是假药,也告诉了 Bob 真药和假药的区别。她想考考 Bob,于是规定了一种称重方式:
	
	\begin{quote}
		从每瓶药中取出任意粒药丸,一并称出取出的药丸的总重量。
		
		形式化地,你可以任意构造一个数组$a_1,a_2......a_n(0\leq a_i\leq n)$,从第$i$瓶中取出$a_i$粒药丸并消耗一次称重次数得到这 $\sum\limits_{i = 1}^{k}a_i$ 粒药丸的总重量。
	\end{quote}
	
	Alice问Bob,在她给定的称重方式下,至少要称重多少次才能确定第几瓶药是假的。但是 Bob 一时想不出来,于是找了你来帮忙。
	
	\subsection*{输入描述}
	第一行一个整数 $T(1 \leq T \leq 10^5)$表示数据组数。
	
	随后$T$行,每行一个整数 $n(1 \leq n \leq 10^9)$,表示药的瓶数 。
	
	\subsection*{输出描述}
	对于每组数据,输出一行一个整数表示最少称重次数。
	
	\subsection*{样例}
	\begin{examplebox}{
		2\\
		1\\
		2
	}{
		0\\
		1
	}
	\end{examplebox}
	
	\subsection*{说明/提示}
	
	如果有两瓶药,我们只从某一瓶中取一粒药丸并称重。若该药丸重510mg,则对应的那瓶药为假;否则,另一瓶为假。称重一次即可,答案为1。
	
\end{problem}