\begin{problem}[像素水果忍者]
	
	年轻的云(又被称为秧歌)最近迷上了一款叫作《水果忍者》的游戏。在这个游戏中,屏幕上会出现水果,你需要在屏幕上滑动,如果滑动的路径和水果有重合的部分,你就能切开这个水果并获得分数。
	
	由于这个游戏只能免费玩114514秒,秧歌决定自己开发一款新游戏《像素水果忍者》。在这里,水果都是矩形的且不会倾斜,而你画出路径都是线段。秧歌想让你帮他完成判断水果是否被切开的功能。
	
	游戏可以抽象为一个平面直角坐标系。每次,给出一条线段的左端点$(x_1, y_1)$ 和右端点 $(x_2, y_2)$,如果线段垂直于$x$轴则按任意顺序给出;给出一个矩形的左下角 $(x_l, y_l)$ 和右上角 $(x_r, y_r)$ ,保证四边平行于坐标轴;判断线段和矩形有没有交集。
	
	\subsection*{输入描述}
	第一行一个整数 $T(1\leq T\leq 10^5)$表示数据组数。接下来按组给出数据。
	
	对于每组数据:
	\begin{quote}
	第一行四个整数 $x_1,y_1,x_2,y_2$表示线段的左右端点。
	
	第二行四个整数 $x_l,y_l,x_r,y_r$表示矩形左下角和右上角。
	\end{quote}
	
	保证给出的坐标值的绝对值不超过 $10^4$。保证线段不会退化成点,矩形不会退化成线段或点。
	
	\subsection*{输出描述}
	对于每组数据,如果有交集则输出一行YES,否则输出一行NO。
	
	\subsection*{样例}
	\begin{examplebox}{
		3\\
		0 0 5 5\\
		0 0 10 10\\
		0 0 5 5\\
		6 6 7 7\\
		0 0 5 5\\
		1 1 2 2
	}{
		YES\\
		NO\\
		YES
	}
	\end{examplebox}
	
	\subsection*{说明/提示}
	如果存在一个点既在线段上(含端点)也在矩形中(含边界),则认为线段和矩形有交集。
\end{problem}