\begin{problem}[水+水=水]
	
	在一个二维网格世界中,水超现实地流动着。网格大小为$m \times n$,左上角为 $(0,0)$,右下角为$(m,n)$。
	
	\begin{quote}
		「水源」:水流动的中心。
		
		「流动」:水会以水源为中心向周围流动,每流动一格,流量-1。流量降至0时,水不再继续流动。
		
		「流量」:一个格子可能会被多处水源流动到,该格子的流量为所有水源流动到此处的流量的最大值。水源的流量为5。
		
		「水的生成」:当一个格子周围四个存在至少两个水源,则该格子变为水源。水源会不断生成,直到没有格子满足生成水源的条件。
	\end{quote}
	
	例如,水流的最终状态可能是这样:
	
	\begin{figure}[H]
		\centering
		\includegraphics[width=0.5\textwidth]{flow.png}
		\caption{水的流动} % 图片标题
		\label{flow} % 标签,用于交叉引用
	\end{figure}
	
	如图\ref{to_merge}所示,被标记的格子周围存在两个水源,因此会形成新的水源。
	\begin{figure}[H]
		\centering
		\includegraphics[width=0.5\textwidth]{to_merge.png}
		\caption{中间状态} % 图片标题
		\label{to_merge} % 标签,用于交叉引用
	\end{figure}
	
	新的水源生成后,最终状态如图\ref{merged}所示。
	\begin{figure}[H]
		\centering
		\includegraphics[width=0.5\textwidth]{merged.png}
		\caption{新的水源生成} % 图片标题
		\label{merged} % 标签,用于交叉引用
	\end{figure}
	
	世界创立之初,k个格子上存在水源,随之,水开始流动、生成。请你计算最终每个格子上的流量。
	
	\subsection*{输入描述}
	第一行一个整数$T(1\leq T\leq 100)$表示数据组数,接下来按组给出数据。
	
	对于每组数据:
	\begin{quote}
		第一行包含三个整数 $m, n, k$($1 \le m, n \le 20$),表示网格的行数和列数、初始水源数量。
		
		接下来 $k$ 行,每行包含两个整数 $x_i, y_i$($0 \le x_i < m,\ 0 \le y_i < n$)表示初始水源的坐标。保证坐标不会重复。
	\end{quote}
	
	\subsection*{输出描述}
	对于每组数据,输出m行,每行包含n个整数,表示最终每个格子的流量。
	
	\subsection*{样例}
	\begin{examplebox}{
		1\\
		4 6 2\\
		0 0\\
		1 1
	}{
		554321\\
		554321\\
		443210\\
		332100
	}
	\end{examplebox}
	
	\subsection*{说明/提示}
	在$(0,1)$和$(1,0)$处生成了新的水源。
	
\end{problem}