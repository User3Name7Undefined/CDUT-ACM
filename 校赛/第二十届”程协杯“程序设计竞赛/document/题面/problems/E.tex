\begin{problem}[古希腊掌管三角形的神]
	
	秧歌 (又被称为年轻的云) 是古希腊掌管三角形的神,无所畏惧的你来到秧歌的面前想要挑战他的权威,秧歌不屑一顾地对你说:“何意味?正确回答这个问题你才有资格触碰神圣的三角形”。
	
	秧歌将 $n$ 根木棍摆在你的面前,第 $i$ 根木棍的长度为 $a_i$。接着他会提出 $q$ 个对木棍的查询,每个查询会给出两个整数 $l,r$ ( $1\leq l< r\leq n$,且 $r - l + 1 \geq 6$ )。你需要回答,能否从第 $l$ 到 $r$ (包含$l,r$)个木棍中,挑选出 $6$ 根不同的木棍,组成 $2$ 个非退化三角形(若一个三角形的三边长 $a、b、c$ 满足 $a < b + c$且$b < a + c$且$c < a + b$,则称其为非退化三角形)。
	
	
	
	\subsection*{输入描述}
	第一行两个整数 $n$ 和 $q$ ( $6\leq n\leq 10^5,1\leq q\leq 10^5$)分别表示木棍的数量和查询的数量。
	
	第二行 $n$ 个整数 $a_1, a_2......a_n$ ( $1\leq a_i\leq 10^9$ )表示木棍长度。
	
	接下来 $q$ 行,每行两个整数 $l,r$ ( $1\leq l < r\leq n$,且 $r - l + 1 \geq 6$ )表示询问。
	
	\subsection*{输出描述}
	对于每个查询,若能组成 $2$ 个非退化三角形,输出一行 YES,否则输出一行 NO。
	
	\subsection*{样例}
	\begin{examplebox}{
		10 5\\
		5 2 2 10 4 10 6 1 5 3\\
		1 6\\
		2 7\\
		2 8\\
		5 10\\
		4 10
	}{
		YES\\
		NO\\
		YES\\
		NO\\
		YES
	}
	\end{examplebox}
	
	
	\subsection*{说明/提示}
	第一个查询中,木棍长度为 $[5, 2, 2, 10, 4, 10]$,可选择两组木棍 $[2, 4, 5]$ 和 $[2, 10, 10]$,分别组成 $2$ 个非退化三角形。
	
	第二个查询中,木棍长度为 $[2, 2, 10, 4, 10, 6]$,无法组成 $2$ 个非退化三角形。
	
	第三个查询中,木棍长度为 $[2, 2, 10, 4, 10, 6, 1]$,可选择两组木棍 $[1, 2, 2]$ 和 $[4, 10, 10]$,分别组成 $2$ 个非退化三角形。
	
	第四个查询中,木棍长度为 $[4, 10, 6, 1, 5, 3]$,无法组成 $2$ 个非退化三角形。
	
	第五个查询中,木棍长度为 $[10, 4, 10, 6, 1, 5, 3]$,可选择两组木棍 $[1, 10, 10]$ 和 $[3, 4, 5]$,分别组成 $2$ 个非退化三角形。
\end{problem}