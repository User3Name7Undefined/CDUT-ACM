\documentclass{beamer}
\usepackage[UTF8]{ctex} % 中文支持
\usepackage{amsmath} % 数学公式
\usepackage{amssymb} % 数学符号

% 主题设置
\usetheme{Madrid}
\usecolortheme{seahorse}

% 标题页信息
\title{第二十届“程协杯”程序设计竞赛题目讲解}
\author{第二十届“程协杯”命题组}
\date{today}

\begin{document}
	
	% 标题页
	\begin{frame}
		\titlepage
	\end{frame}
	
	% 目录页(可选)
	\begin{frame}{目录}
		\tableofcontents
	\end{frame}
	
	% 题目难度分布页
	\begin{frame}{难度分布}
		
		Easy:B、G
		
		Easy-Medium:F、J、K、M
		
		Medium:A、D、I
		
		Medium-Hard:C、E、H
		
		Hard:L
		
	\end{frame}
	
	\section{B}
	\begin{frame}{B.罚时}
		\begin{block}{分析}
			设最终解决问题的次序为a,b,c.....则总用时为$(t_a)+(t_a+t_b)+(t_a+t_b+t_c)+......$。
			
			先解决的题目会被计算更多次,因此需要先解决花费时间更少的题目。
		\end{block}
		
		\begin{block}{解法}
			把题目按照花费时间从小到大排序,设$T$为题目用时,则$T_i\leftarrow T_{i-1}+t_i$。
			
			求出$\sum_{i=1}^n T$即可。
		\end{block}
	\end{frame}
	
	\section{G}
	\begin{frame}{G.恋爱物语}
		\begin{block}{解法}
			判断是否有$a[b[i]]=i$即可。
		\end{block}
	\end{frame}
	
\end{document}